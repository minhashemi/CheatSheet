%by Amin Hashemi (@minhashemi)
%	Spring 2023

\documentclass{article}
\usepackage[landscape, a4paper]{geometry}
\usepackage{url}
\usepackage{multicol}
\usepackage{amsmath}
\usepackage{esint}
\usepackage{amsfonts}
\usepackage{tikz}
\usepackage{multirow}
\usetikzlibrary{decorations.pathmorphing}
\usepackage{amsmath,amssymb}
\usepackage{tabularx} % for adjusting column widths
\usepackage{booktabs}
\usepackage[makeroom]{cancel}
\usepackage{colortbl}
\usepackage{xcolor}
\usepackage{mathtools}
\usepackage{amsmath,amssymb}
\usepackage{enumitem}
\makeatletter

\newcommand*\bigcdot{\mathpalette\bigcdot@{.5}}
\newcommand*\bigcdot@[2]{\mathbin{\vcenter{\hbox{\scalebox{#2}{$\m@th#1\bullet$}}}}}
\makeatother

\title{Statistics Cheat Sheet}
\usepackage[brazilian]{babel}
\usepackage[utf8]{inputenc}

\advance\topmargin-.8in
\advance\textheight3in
\advance\textwidth3in
\advance\oddsidemargin-1.45in
\advance\evensidemargin-1.45in
\parindent0pt
\parskip2pt
\newcommand{\hr}{\centerline{\rule{3.5in}{1pt}}}
%\colorbox[HTML]{e4e4e4}{\makebox[\textwidth-2\fboxsep][l]{texto}
\usepackage{fontspec}

\usepackage{polyglossia}
\setmainfont[Path=/Users/aminhashemi/Library/Fonts/]{IRANSans.ttf}
\setdefaultlanguage{farsi}
%\setotherlanguages{english}


\begin{document}



































































































































\begin{multicols*}{2}

\tikzstyle{mybox} = [draw=black, fill=white, very thick,
    rectangle, rounded corners, inner sep=10pt, inner ysep=10pt]
\tikzstyle{fancytitle} =[fill=black, text=white, font=\bfseries]





%------------ Measures of Spread ---------------
\begin{tikzpicture}
\node [mybox] (box){%
    \begin{minipage}{0.43\textwidth}
  
 \begin{RTL}
اگر $L_i(x)$ چندجمله‌ای درجه $n$ باشد، چند‌جمله‌ای درونیاب برابر است با: (درجه $P(x) $ بعد از محاسبات معلوم می‌شود)
$$P(x) = \sum_{i=o}^n L_i(x)f_i$$
که
$$L_j(x) = \frac{(x-x_0)(x-x_1)\cdots{\color{blue} (x-x_{j-1})(x-x_{j+1})}\cdots (x-x_n)}{(x_j-x_0)(x_j-x_1)\cdots{\color{blue} (x_j-x_{j-1})(x_j-x_{j+1})}\cdots (x_j-x_n)}$$

\end{RTL}
    
    
    \end{minipage}
};
%------------ Measures of Variability Header ---------------------
\node[fancytitle, left=10pt, rounded corners=4pt, draw=black,fill=white, text=black] at (box.north east) {$\text{‌درونیابی لاگرانژ}$};
\end{tikzpicture}


%------------ Discrete Random Variable and Distributions ---------------
\begin{tikzpicture}
\node [mybox] (box){%
    \begin{minipage}{0.43\textwidth}
  \begin{RTL}
  \def\arraystretch{1.5}%  1 is the default, change whatever you need
  \begin{tabular}{ l r}
مرتبه اول:
&
$f[x_i, x_{i+1}] = \frac{f_i - f_{i+1}}{x_i - x_{i+1}}$
\\
مرتبه دوم:
&
$f[x_i, x_{i+1},x_{i+2}] = \frac{f[x_i, x_{i+1}]  - f[x_{i+1}, x_{i+2}] }{x_i - x_{i+2}} $
\end{tabular}

چندجمله‌ای درونیاب:

\begin{align*}
P(x) = f_0 + (x-x_0)f[x_0,x_1] &+ (x-x_0)(x-x_1)f[x_0,x_1,x_2] + \cdots  \\&+ (x-x_0)(x-x_1)\cdots(x-x_{n-1})f[x_0,x_1,\dots, x_n]
\end{align*}


\end{RTL}
    
    \[
\begin{array}{c|cccc}
x_i & f[x_i] & f[x_i,x_{i+1}] & f[x_i,x_{i+1},x_{i+2}] & \cdots \\
\hline
x_0 & f[x_0] \\
x_1 &  \color{gray}f[x_1] & f[x_0,x_1] \\
x_2 &  \color{gray}f[x_2] &  \color{gray}f[x_1,x_2] & f[x_0,x_1,x_2] \\
\vdots & \vdots & \vdots & \vdots & \ddots \\
x_n &  \color{gray}f[x_n] &  \color{gray}f[x_{n-1},x_n] &  \color{gray}f[x_{n-2},x_{n-1},x_n] & \cdots f[x_0,x_1,\ldots,x_n]
\end{array}
\]

    \end{minipage}
};
%------------ Discrete Random Variable and Distributions Header ---------------------
\node[fancytitle, left=10pt, rounded corners=4pt, draw=black,fill=white, text=black] at (box.north east) {$\text{‌درونیابی تفاضلات تقسیم شده نیوتن}$};
\end{tikzpicture}


%--------------
\begin{tikzpicture}
\node [mybox] (box){%
    \begin{minipage}{0.43\textwidth}


    	\begin{RTL}
در دستگاه غیر خطی $[f(x,y) = 0 , g(x,y) = 0]$،  $(x_i,y_i)$ تخمین جوابِ  طوری که:
$$
x_{n+1} = x_n + h_n \qquad
y_{n+1} = y_n + k_n $$
با فرض معلوم بودن $x_n, y_n$:
$$\begin{cases}
h_n \frac{\partial f}{\partial x} \big|_{(x_n,y_n)} + k_n \frac{\partial f}{\partial y} \big|_{(x_n,y_n)} = -f(x_n,y_n) \\
h_n \frac{\partial g}{\partial x} \big|_{(x_n,y_n)} + k_n \frac{\partial g}{\partial y} \big|_{(x_n,y_n)} = -g(x_n,y_n)
\end{cases}$$
در این صورت:



\begin{center}
\begin{tabular}{c c c}
$h_n = \frac{\begin{vmatrix}
-f & f_y\\
-g & g_y
\end{vmatrix}}{\begin{vmatrix}
f_x & f_y\\
g_x & g_y
\end{vmatrix}}$
& $k_n = \frac{\begin{vmatrix}
f_x & -f\\
g_x & -g
\end{vmatrix}}{\begin{vmatrix}
f_x & f_y\\
g_x & g_y
\end{vmatrix}}$
&
{\color{gray!80}حواست باشه مخرج صفر نشه}
\end{tabular}
\end{center}

	\end{RTL}
	
    \end{minipage}
};
%------------ Measures of Center Header ---------------------
\node[fancytitle, left=10pt, rounded corners=4pt, draw=black,fill=white, text=black] at (box.north east) {$\text{دستگاه معادلات غیر خطی}$};
\end{tikzpicture}


%--------------

%------------ Discrete Random Variable and Distributions ---------------
\begin{tikzpicture}
\node [mybox] (box){%
    \begin{minipage}{0.43\textwidth}
      \begin{RTL}
مخصوص درونیابی نقاط با فاصله مساوی

تغییر متغیر:
$$f(x) = f(x_0+sh) = f_s$$
\underline{تفاضل پیشرو}
\begin{align*}
\Delta^i f_s = \begin{cases}
f_s & i=0 \\
\Delta^{i-1} f_{s+1} - \Delta^{i-1} f_{s} & i>0
\end{cases}
\end{align*}
چندجمله‌ای درونیاب
$$P(x) = f_0 + {s\choose 1}\Delta f_0 + {s\choose 2}\Delta^2 f_0 + \cdots + {s\choose n}\Delta^n f_0$$
جاگذاری $ s= (x-x_0)\big/h$ و محاسبه تابع بر حسب $x$.
\end{RTL}

    \[
\begin{array}{c|ccccc}
x_i & f_i& \Delta& \Delta^2 & \Delta^3 & \cdots \\
\hline
x_0 & f_0 \\
x_1 &  \color{gray}f_1 & \Delta f_0 \\
x_2 &  \color{gray}f_2 & \color{gray} \Delta f_1 & \Delta^2 f_0\\
x_3 &  \color{gray}f_3 &  \color{gray}\Delta f_2 &  \color{gray}\Delta^2 f_1 & \Delta^3 f_0 \\
\end{array}
\]

    \end{minipage}
};
%------------ Discrete Random Variable and Distributions Header ---------------------
\node[fancytitle, left=10pt, rounded corners=4pt, draw=black,fill=white, text=black] at (box.north east) {$\text{‌تفاضل پیشرو}$};
\end{tikzpicture}


%------------ Discrete Random Variable and Distributions ---------------
\begin{tikzpicture}
\node [mybox] (box){%
    \begin{minipage}{0.43\textwidth}
      \begin{RTL}
مخصوص برای $x$ نزدیک به نقاط انتهایی جدول.

تغییر متغیر:
$$f(x) = f(x_n+sh) = f_s$$
\underline{تفاضل پسرو}
\begin{align*}
\nabla^i f_s = \begin{cases}
f_s & i=0 \\
\nabla^{i-1} f_{s} - \nabla^{i-1} f_{s-1} & i>0
\end{cases}
\end{align*}
چندجمله‌ای درونیاب
\begin{align*}
P(x) = f_n +s\nabla f_n + \frac{s(s+1)}{2!}\nabla^2 f_n &+\cdots \\
&+\frac{s(s+1)(s+2)\cdots (s+n-1)}{n!}\nabla^n f_n
\end{align*}

جاگذاری $ s= (x-x_0)\big/h$ و محاسبه تابع بر حسب $x$.
\end{RTL}

    \[
\begin{array}{c|ccccc}
x_i & f_i& \Delta& \Delta^2 & \Delta^3 & \cdots \\
\hline
x_0 & \color{gray}f_0 & \color{gray}\nabla f_1 &\color{gray} \nabla^2 f_2 & \nabla^3 f_3\\
x_1 & \color{gray}f_1& \color{gray}\nabla f_2 & \nabla^2 f_3 \\
x_2 & \color{gray}f_2&  \nabla f_3  \\
x_3 &  f_3  \\
\end{array}
\]

    \end{minipage}
};
%------------ Discrete Random Variable and Distributions Header ---------------------
\node[fancytitle, left=10pt, rounded corners=4pt, draw=black,fill=white, text=black] at (box.north east) {$\text{‌تفاضل پسرو}$};
\end{tikzpicture}






\begin{tikzpicture}
\node [mybox] (box){%
    \begin{minipage}{0.46\textwidth}
    
    	\begin{RTL}
تابع درونیاب را پیدا می‌کنیم و مشتق می‌گیریم.

\def\arraystretch{1.5}%  1 is the default, change whatever you need
\begin{tabular}{l l  r}
$f'_i = \frac{1}{h} \Delta f_i$&$f_i' = \frac{f_{i+1}-f_i}{h}$ & مشتق مرتبه اول با یک جمله  \\
$f'_i = \frac{1}{h}[\Delta f_i - \frac{1}{2}\Delta^2 f_i]$&$f_i' = \frac{2f_{i+1} - \frac{1}{2} f_{i+2} - \frac{3}{2} f_i}{h}$ & مشتق مرتبه اول با دو جمله \\
$f''_i = \frac{1}{h^2} \Delta^2 f_i$&$f_i'' = \frac{f_{i+2}-2f_{i+1}+f_i}{h^2}$ & مشتق مرتبه دوم با دو جمله
\end{tabular}

	\end{RTL}
	
    \end{minipage}
};
%------------ Measures of Center Header ---------------------
\node[fancytitle, left=10pt, rounded corners=4pt, draw=black,fill=white, text=black] at (box.north east) {$\text{‌مشتق‌گیری عددی تابع درونیاب}$};
\end{tikzpicture}


\begin{tikzpicture}
\node [mybox] (box){%
    \begin{minipage}{0.46\textwidth}
    
    	\begin{RTL}
بسط تیلور $f(x_{i+1})$ و $f(x_{i-1})$ را می‌نویسیم:

\def\arraystretch{1.5}%  1 is the default, change whatever you need
\begin{tabular}{l  r}
$f'(x_i) = \frac{f_{i+1}-f_i}{h}$ & مشتق مرتبه اول با دو جمله  در $f'(x_0)$ \\
$f'(x_i) = \frac{f_{i}-f_{i-1}}{h}$  & مشتق مرتبه اول با دو جمله در $f'(x_n)$ \\
$f'(x_i) = \frac{f_{i+1}-f_{i-1}}{2h}$ & در حالت کلی \\
$f''(x_i) = \frac{f_{i-1} - 2f_i + f_{i+1}}{h^2}$ & مشتق مرتبه دوم \\
$f'''(x_i) = \frac{f_{i+2} - 2f_{i+1}+2f_{i-1}-f_{i-2}}{2h^3}$ & مشتق مرتبه سوم
\end{tabular}

	\end{RTL}
	
    \end{minipage}
};
%------------ Measures of Center Header ---------------------
\node[fancytitle, left=10pt, rounded corners=4pt, draw=black,fill=white, text=black] at (box.north east) {$\text{‌مشتق‌گیری بسط تیلور}$};
\end{tikzpicture}

\begin{tikzpicture}
\node [mybox] (box){%
    \begin{minipage}{0.46\textwidth}
    
    	\begin{RTL}
    	بین زیر‌بازه‌ها، تابع خطی رد کن
\begin{align*}
\int_{x_i}^{x_{i+1}} f(x) dx = \frac{h}{2} (f_i + f_{i+1}) \\
\Rightarrow T(h) = \int_a^b f(x) dx &= \frac{h}{2} [f_0 + 2f_1 + \cdots + 2f_{n-1} + f_n] \\
&= A_0 f_0 + A_1f_1 + \cdots + A_nf_n + E, \quad A_i = A_{i+1} = \frac{h}{2}
\end{align*}


	\end{RTL}
	
    \end{minipage}
};
%------------ Measures of Center Header ---------------------
\node[fancytitle, left=10pt, rounded corners=4pt, draw=black,fill=white, text=black] at (box.north east) {$\text{‌‌انتگرال ذوزنقه‌ای}$};
\end{tikzpicture}

\begin{tikzpicture}
\node [mybox] (box){%
    \begin{minipage}{0.46\textwidth}
    
    	\begin{RTL}
    	بین زیر‌بازه‌ها تابع درجه ۲ رد کن. (تعداد نقاط باید فرد باشد.)
\begin{align*}
\int_{x_i}^{x_{i+2}} f(x) dx &= \frac{h}{3} (f_i + 4f_{i+1} + f_{i+2}) \\
\Rightarrow S(h) = \int_{x_0}^{x_n} f(x) dx &= \frac{h}{3} [f_0 + 4f_1 + 2f_2 + 4f_3 + 2f_4 +  \cdots + 4f_{n-1} + f_n] \\
&= A_0 f_0 + A_1f_1 + \cdots + A_nf_n + E
\end{align*}


	\end{RTL}
	
    \end{minipage}
};
%------------ Measures of Center Header ---------------------
\node[fancytitle, left=10pt, rounded corners=4pt, draw=black,fill=white, text=black] at (box.north east) {$\text{‌‌انتگرال سیمپسون}$};
\end{tikzpicture}



\begin{tikzpicture}
\node [mybox] (box){%
    \begin{minipage}{0.46\textwidth}
    
    	\begin{RTL}

برای توابع تکین (تعریف نشده در ابتدا یا انتها).
$\int_{x_i}^{x_{i+1}} f(x) dx = hf\left(x_i + \frac{h}{2}\right) $

\begin{align*}
M(h) = \int_{x_0}^{x_n} f(x) dx =   h\left[f\left(x_0 + \frac{h}{2}\right) +\cdots + f\left(x_i + \frac{h}{2}\right)+ \cdots+ f\left(x_{n-1} + \frac{h}{2}\right)\right]
\end{align*}

	\end{RTL}
	
    \end{minipage}
};
%------------ Measures of Center Header ---------------------
\node[fancytitle, left=10pt, rounded corners=4pt, draw=black,fill=white, text=black] at (box.north east) {$\text{‌‌انتگرال نقطه میانی}$};
\end{tikzpicture}

\begin{tikzpicture}
\node [mybox] (box){%
    \begin{minipage}{0.46\textwidth}
    
    	\begin{RTL}
$x_i$ها معلوم

برای $f(x) = 1, x, x^2, \cdots$، خطای سیمپسون $0=$

$A_i$ها را تعیین می‌کنیم.
    	
    	در این روش، تعداد نقاط باید \underline{فرد} باشد.


\underline{نیوتن-کاتس ۴ نقطه‌ای}
\begin{align*}
\int_0^{3h} f(x) dx = \frac{3h}{8} [f_0+3f_1+3f_2+f_3]
\end{align*}


	\end{RTL}
	
    \end{minipage}
};
%------------ Measures of Center Header ---------------------
\node[fancytitle, left=10pt, rounded corners=4pt, draw=black,fill=white, text=black] at (box.north east) {$\text{‌‌انتگرال نیوتن-کاتس(ضرایب مجهول)}$};
\end{tikzpicture}

\begin{tikzpicture}
\node [mybox] (box){%
    \begin{minipage}{0.46\textwidth}
    
    	\begin{RTL}
در سیمپسون:
\begin{itemize}
\item  $A_i, x_i$ مجهول
\item خطا $0=$
\end{itemize}

فرمول‌ها برای $[-1,1]$ بدست می‌آید.

نقاط زیر‌بازه لزوما متساوی الفاصله نیستند.

\begin{align*}
\begin{cases}
x \in [a,b] \\
u \in [-1,1]
\end{cases}
\to x = \frac{1}{2} [(b-a)u + (b+a)] \Rightarrow  \int_a^b g(u) \, du = \frac{b-a}{2}\int_{-1}^1 f(x)\,dx
\end{align*}

\underline{گاوس ۲ نقطه‌ای}

$$\int_{-1}^1 f(x) \, dx = f\left(\frac{-\sqrt{3}}{3}\right) + f\left(\frac{\sqrt{3}}{3}\right)$$


\underline{گاوس ۳ نقطه‌ای}

$$\int_{-1}^1 f(x) \, dx = \frac{1}{9}\left[5f\left(-\sqrt{\frac{3}{5}}\right) + 8f(0) + 5f\left(\sqrt{\frac{3}{5}}\right) \right]$$

	\end{RTL}
	
    \end{minipage}
};
%------------ Measures of Center Header ---------------------
\node[fancytitle, left=10pt, rounded corners=4pt, draw=black,fill=white, text=black] at (box.north east) {$\text{‌‌انتگرال گاوس}$};
\end{tikzpicture}



\begin{tikzpicture}
\node [mybox] (box){%
    \begin{minipage}{0.46\textwidth}
    
    	\begin{RTL}

 رگرسیون خطی :)
\begin{align*}
\underset{a_0, a_1, \cdots}{\mathrm{argmin}}  \sum_{i=1}^n (y_i - \bar{y})^2 = \sum_{i=1}^n \left(y_i - (a_0 + a_1x + \cdots)\right)^2 = s \Rightarrow \begin{cases} \frac{\partial s}{\partial a_0}= 0 \\ \frac{\partial s}{\partial a_1} = 0 \\ \vdots\end{cases}
\end{align*}

$$\begin{cases}
\cancelto{n}{(\sum 1)} a_0 +( \sum x_i) a_1 + \cdots = \sum y_i \\
(\sum x_i) a_0 + (\sum x_i^2) a_1 +\cdots= \sum x_i y_i \\
\vdots
\end{cases}$$
	\end{RTL}
	
    \end{minipage}
};
%------------ Measures of Center Header ---------------------
\node[fancytitle, left=10pt, rounded corners=4pt, draw=black,fill=white, text=black] at (box.north east) {$\text{‌روش حداقل مربعات}$};
\end{tikzpicture}




\begin{tikzpicture}
\node [mybox] (box){%
    \begin{minipage}{0.46\textwidth}
    
    	\begin{RTL}

بدون در نظر گرفتن خطای گرد کردن، به جواب دقیق می‌رسیم.

عملیات سطری مقدماتی:
\begin{itemize}
\item
جابجایی سطر‌ها
\item
ضرب سطر در عدد غیر صفر
\item
جمع دو سطر با یکدیگر
\end{itemize}

\begin{tabular}{r r}
جایگذاری پسرو & حل به کمک ماتریس بالا مثلثی \\
جایگذاری پیشرو & حل به کمک ماتریس پایین مثلثی
\end{tabular}

{\large\underline{الگوریتم}}
\begin{enumerate}
\item محور‌گیری

\begin{itemize}
\item 
با جابجایی سطر‌ها، همه اعداد بزرگ را روی قطر انتقال بده. 
\item 
سپس بالا مثلثی کن  (پسرو) یا پایین مثلثی کن (پیشرو). 

\color{gray}\item
{\color{gray} علت: کاهش خطای حاصل ضرب }

\end{itemize}
\item 
انجام عملیات سطری مقدماتی روی ماتریس افزوده  
$\begin{pmatrix}
A | b
\end{pmatrix}$

\item

تکرار ۲، تا رسیدن به بالامثلثی (پسرو) یا پایین مثلثی (پیشرو)
\item
برای انجام مرحله $i$-اُم، باید $a_{ii}^{(i)} \neq 0$. (عضو روی قطر، غیر صفر)
\end{enumerate}

	\end{RTL}
	
    \end{minipage}
};
%------------ Measures of Center Header ---------------------
\node[fancytitle, left=10pt, rounded corners=4pt, draw=black,fill=white, text=black] at (box.north east) {$\text{‌روش حذفی گاوس - مستقیم}$};
\end{tikzpicture}


\begin{tikzpicture}
\node [mybox] (box){%
    \begin{minipage}{0.46\textwidth}
    
    	\begin{RTL}
به تقریب جواب می‌رسیم.

\begin{enumerate}
\item 
از معادله $i$-اُم، $x_i$ را بدست می‌آوریم.
($i.e.\,\,\, x_i = \cdots$)
\item
تقریب اولیه 
$x^{(0)} = [x_1^{(0)}, x_2^{(0)}, x_3^{(0)}, \cdots]^t$
را در نظر می‌گیریم.
\item
با جاگذاری تقریب ۲ در دستگاه ۱، یک تقریب جدید بدست می آید. (تکرار شود...)\
\begin{align*}
x_1^{(k+1)} = 1\big/a_{11}[b_1 - a_{12}x_2^{(k)} - a_{13}x_3^{(k)} -\cdots] \\
x_2^{(k+1)} = 1\big/a_{22}[b_2 - a_{21}x_1^{(k)} - a_{23}x_3^{(k)} -\cdots] \\
x_3^{(k+1)} = 1\big/a_{33}[b_3 - a_{31}x_1^{(k)} - a_{32}x_2^{(k)} -\cdots]
\end{align*}
\end{enumerate}
	\end{RTL}
	
    \end{minipage}
};
%------------ Measures of Center Header ---------------------
\node[fancytitle, left=10pt, rounded corners=4pt, draw=black,fill=white, text=black] at (box.north east) {$\text{روش ژاکوبی - تکراری}$};
\end{tikzpicture}



\begin{tikzpicture}
\node [mybox] (box){%
    \begin{minipage}{0.46\textwidth}
    
    	\begin{RTL}

بدون در نظر گرفتن خطای گرد کردن، به جواب دقیق می‌رسیم.

$$x = A^{-1} b$$

\begin{itemize}
\item
اگر $b=0, |A| \neq 0$، آنگاه $x=0$
\item
اگر $|A|=0$، آنگاه دستگاه همگن $Ax=0$ جواب غیر صفر دارد.
\end{itemize}


\underline{محاسبه ماتریس وارون}

ماتریس افزوده 
$\begin{pmatrix}
A | I
\end{pmatrix}$
را با عملیات سطری مقدماتی به 
$\begin{pmatrix}
I | A^{-1}
\end{pmatrix}$
تبدیل می‌کنیم.
\\ \\
\underline{rank}
ابعاد بزرگترین زیرماتریس مربعی که دترمینان آن مخالف صفر باشد. (تعداد سطر‌های مستقل خطی)

ماتریسی وارون پذیر است که $Order(A) = Rank(A)$
\\ \\
\underline{حل پذیری دستگاه}

\begin{itemize}
\item 
اگر $Rank(A) \neq Rank (A|Y)$ دستگاه پاسخ ندارد
\item
اگر $Rank(A) = Rank(A|Y) = r$، دستگاه سازگار است
\begin{itemize}
\item
$r<n$ : بی‌نهایت پاسخ از خانواده $n-r$ پارامتر
\item $n=r$ : پاسخ یکتا
\end{itemize}
\end{itemize}
	\end{RTL}
	
    \end{minipage}
};
%------------ Measures of Center Header ---------------------
\node[fancytitle, left=10pt, rounded corners=4pt, draw=black,fill=white, text=black] at (box.north east) {$\text{‌روش ماتریس وارون ضرایب  - مستقیم}$};
\end{tikzpicture}



\begin{tikzpicture}
\node [mybox] (box){%
    \begin{minipage}{0.46\textwidth}
    
    	\begin{RTL}
مثل ژاکوبی

در هر مرحله، محورگیری انجام بده.

تفاوت:

هر مقدار جدید بدست آمده از مولفه‌های جواب، در محاسبه سایر مولفه‌ها بکار می‌رود.

\begin{LTR}
\begin{align*}
x_1^{(k+1)} = 1\big/a_{11}[b_1 - a_{12}x_2^{(k)} - a_{13}x_3^{(k)} -\cdots] \\
x_2^{(k+1)} = 1\big/a_{22}[b_2 - a_{21}x_1^{(k+1)} - a_{23}x_3^{(k)} -\cdots] \\
x_3^{(k+1)} = 1\big/a_{33}[b_3 - a_{31}x_1^{(k+1)} - a_{32}x_2^{(k+1)} -\cdots]
\end{align*}
\end{LTR}

گاوس-سایدل، سریعتر از ژاکوبی. (هیچکدام تضمین همگرایی ندارد)
\\ \\
اگر ماتریس $A$ (ضرایب)، اکیدا قطر غالب سطری/ستونی باشد، هر حدس اولیه از جواب‌ها برای ژاکوبی و گاوس-سایدل، همگراست


	\end{RTL}
	
    \end{minipage}
};
%------------ Measures of Center Header ---------------------
\node[fancytitle, left=10pt, rounded corners=4pt, draw=black,fill=white, text=black] at (box.north east) {$\text{روش گاوس-سایدل - تکراری}$};
\end{tikzpicture}










\begin{flushleft}
\begin{tabular}{c  c}
 \textbf{محاسبات عـــددی} & بهار ۱۴۰۲ \\
 دکتر کیوان محمدی &  \raisebox{-.5\height}{\includegraphics[scale=0.1]{amin handwritten.ai}}
\end{tabular}

\end{flushleft}



\end{multicols*}
\end{document}
